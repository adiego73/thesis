%*******************************************************
% Abstract
%*******************************************************
% !TeX spellcheck = en_US
%\renewcommand{\abstractname}{Abstract}

\begingroup
\let\clearpage\relax
\let\cleardoublepage\relax
\let\cleardoublepage\relax
\phantomsection
\addcontentsline{toc}{chapter}{\abstractname}

\pdfbookmark[1]{Abstract}{Abstract}
\chapter*{Abstract}

In the context of the Leonardo Drone Contest, where specific constraints and characteristics apply, an implementation of a localization and mapping algorithm is proposed. The drone's constraints include the prohibition of usage of GNSS or laser devices, which enforces the localization algorithm to be robust enough. Moreover, the characteristics of the environment make it possible to use a visual-based localization algorithm. \\

On the other hand, the contest requires to map specific landmarks in the environment in order to be able to follow a sequence of take-off and landing, in a particular order, for those landmarks mapped. In this sense, the contest forces the algorithm to map these landmarks so their coordinates can be used later on.\\

The current work proposes an implementation of an EKF-SLAM algorithm, and tries to shed some light on the usage of these landmarks, while tries to identify implementation details that increase the performance and robustness of the proposed algorithm.

\vfill
\newpage
\pdfbookmark[1]{Sommario}{Sommario}
\chapter*{Sommario}
\begin{otherlanguage}{italian}

Nell'ambito del Leonardo Drone Contest, dove si applicano vincoli e caratteristiche specifiche, viene proposta l'implementazione di un algoritmo di localizzazione e mappatura. I vincoli del drone includono il divieto di utilizzo di dispositivi GNSS o laser, che impone che l'algoritmo di localizzazione sia abbastanza robusto. Inoltre, le caratteristiche dell'ambiente consentono di utilizzare un algoritmo di localizzazione visuale. \\

D'altra parte, il concorso richiede di mappare punti di riferimento specifici nell'ambiente in modo da poter seguire una sequenza di decollo e atterraggio, in un ordine particolare, per quei punti di riferimento mappati. In questo senso, il concorso obbliga l'algoritmo a mappare questi punti di riferimento in modo che le loro coordinate possano essere utilizzate in seguito. \\

Il lavoro attuale propone un'implementazione di un algoritmo EKF-SLAM e cerca di comprendere l'importanza dell'uso di questi punti di riferimento, mentre cerca di identificare i dettagli di implementazione che aumentano le prestazioni e la robustezza dell'algoritmo proposto.

\end{otherlanguage}
\endgroup